\documentclass[a4paper]{article}\usepackage{fullpage}\usepackage[german,ngerman]{babel}\usepackage[utf8]{inputenc}\usepackage{graphicx}\pagenumbering{gobble}\begin{document}~\\\textbf{{\large{}Donnerstag 14:50 bis 16:20}}\nopagebreak \\\begin{tabular} {|p{5cm}|p{4cm}|p{4cm}|p{2cm}|}\hline \textbf{Thema}&\textbf{Betreuer}&\textbf{Raum}&\textbf{Anzahl}\\\hline Spezielle Funktionen und Vektorrechnung&Lilith Diringer&MW 0234&4\\\hline Einführung ins Differenzieren&Ilja Göthel&MW 1050&5\\\hline Experimentieren und Auswerten&Ann-Kathrin Raab&CH 26411&17\\\hline Thermodynamik 1&Maximilian Marienhagen&MW 1250&14\\\hline Quanten- und Atomphysik I&Ismail Achmed-Zade&CH 26410&25\\\hline Elektrische Schaltungen&Christopher Pfeiffer&MW 0250&21\\\hline \end{tabular}\\\vspace{5.00000mm}~\\\textbf{{\large{}Donnerstag 16:40 bis 18:10}}\nopagebreak \\\begin{tabular} {|p{5cm}|p{4cm}|p{4cm}|p{2cm}|}\hline \textbf{Thema}&\textbf{Betreuer}&\textbf{Raum}&\textbf{Anzahl}\\\hline Experimentieren und Auswerten&Ann-Kathrin Raab&CH 26411&16\\\hline Näherungsmethoden&Ilja Göthel&MW 0234&13\\\hline Einführung ins Integrieren&Johannes Rothe&MW 1050&13\\\hline Klassische Mechanik&Maximilian Marienhagen&MW 1250&11\\\hline Himmelsmechanik&Lars Dehlwes&MW 2050&17\\\hline Geometrische Optik&Christopher Pfeiffer&MW 0250&16\\\hline \end{tabular}\\\vspace{5.00000mm}~\\\textbf{{\large{}Freitag 09:00 bis 13:00}}\nopagebreak \\\begin{tabular} {|p{5cm}|p{4cm}|p{4cm}|p{2cm}|}\hline \textbf{Thema}&\textbf{Betreuer}&\textbf{Raum}&\textbf{Anzahl}\\\hline Besichtigung der Forschungsneutronenquelle FRM II&Felix Wechsler&Exkursion FRM II&22\\\hline \end{tabular}\\\vspace{5.00000mm}~\\\textbf{{\large{}Freitag 14:20 bis 15:50}}\nopagebreak \\\begin{tabular} {|p{5cm}|p{4cm}|p{4cm}|p{2cm}|}\hline \textbf{Thema}&\textbf{Betreuer}&\textbf{Raum}&\textbf{Anzahl}\\\hline Spezielle Funktionen und Vektorrechnung&Lilith Diringer&MW 0234&5\\\hline Einführung ins Differenzieren&Ann-Kathrin Raab&CH 26411&5\\\hline Rotationsbewegungen&Vincent Grande&MW 1050&3\\\hline Aufgabenseminar Wärmelehre&Maximilian Marienhagen&MW 2050&11\\\hline Kernphysik&Johannes Rothe&MW 0250&15\\\hline Gewöhnliche Differentialgleichungen&Sven Jandura&MW 1250&23\\\hline Experiment Magnetismus&Lars Dehlwes&Praktikum Magnetismus&6\\\hline Experiment spezifische Elektronenladung&Felix Wechsler&Praktikum spezifische Elektronenladung&6\\\hline Experiment Oszilloskop&Christopher Pfeiffer&Praktikum Oszilloskop&6\\\hline Experiment Brückenschaltung&Martin Großhauser&Praktikum Brückenschaltung&6\\\hline \end{tabular}\\\vspace{5.00000mm}~\\\textbf{{\large{}Freitag 16:10 bis 17:40}}\nopagebreak \\\begin{tabular} {|p{5cm}|p{4cm}|p{4cm}|p{2cm}|}\hline \textbf{Thema}&\textbf{Betreuer}&\textbf{Raum}&\textbf{Anzahl}\\\hline Experimentieren und Auswerten&Ann-Kathrin Raab&CH 26411&3\\\hline Komplexe Wechselstromrechnung&Vincent Grande&MW 1050&11\\\hline Spezielle Relativitätstheorie&Johannes Rothe&MW 1250&16\\\hline Harmonische Schwingungen&Ilja Göthel&MW 2050&3\\\hline Aufgabenseminar klassische Mechanik&Aaron Wild&CH 26410&4\\\hline Experiment Magnetismus&Lars Dehlwes&Praktikum Magnetismus&6\\\hline Experiment spezifische Elektronenladung&Felix Wechsler&Praktikum spezifische Elektronenladung&6\\\hline Experiment Oszilloskop&Christopher Pfeiffer&Praktikum Oszilloskop&6\\\hline Experiment Brückenschaltung&Martin Großhauser&Praktikum Brückenschaltung&6\\\hline Näherungsmethoden&Vincent Grande&CH 26410&12\\\hline Einführung ins Integrieren&Felix Wechsler&MW 2050&13\\\hline \end{tabular}\\\vspace{5.00000mm}~\\\textbf{{\large{}Samstag 09:00 bis 10:30}}\nopagebreak \\\begin{tabular} {|p{5cm}|p{4cm}|p{4cm}|p{2cm}|}\hline \textbf{Thema}&\textbf{Betreuer}&\textbf{Raum}&\textbf{Anzahl}\\\hline Experimentieren und Auswerten&Ann-Kathrin Raab&CH 26411&19\\\hline Thermodynamik 2 - Statistische Physik&Vitaly Andreev&MW 1050&17\\\hline Elektrodynamik 1&Maximilian Keitel&MW 1250&21\\\hline Aufgabenseminar klassische Mechanik&Maximilian Marienhagen&MW 0234&4\\\hline Harmonische Schwingungen&Ilja Göthel&MW 0250&9\\\hline Experiment Brennstoffzelle&Aaron Wild&Praktikum Brennstoffzelle&6\\\hline Experiment Millikan-Versuch&Samuel Moll&Praktikum Millikan-Versuch&6\\\hline Experiment Pohlsches Rad&Eugen Dizer&Praktikum Pohlsches Rad&4\\\hline \end{tabular}\\\vspace{5.00000mm}~\\\textbf{{\large{}Samstag 10:50 bis 12:20}}\nopagebreak \\\begin{tabular} {|p{5cm}|p{4cm}|p{4cm}|p{2cm}|}\hline \textbf{Thema}&\textbf{Betreuer}&\textbf{Raum}&\textbf{Anzahl}\\\hline Experiment Reversionspendel&Lilith Diringer&Praktikum Reversionspendel&6\\\hline Spezielle Relativitätstheorie&Johannes Rothe&MW 0250&12\\\hline Klassische Mechanik&Maximilian Marienhagen&MW 1050&3\\\hline Geometrische Optik&Christopher Pfeiffer&MW 0234&3\\\hline Elektrodynamik 2&Maximilian Keitel&MW 1250&19\\\hline Quanten- und Atomphysik I&Vitaly Andreev&MW 2050&15\\\hline Himmelsmechanik&Lars Dehlwes&CH 26410&6\\\hline Experiment Brennstoffzelle&Aaron Wild&Praktikum Brennstoffzelle&6\\\hline Experiment Millikan-Versuch&Samuel Moll&Praktikum Millikan-Versuch&6\\\hline Experiment Pohlsches Rad&Eugen Dizer&Praktikum Pohlsches Rad&4\\\hline Experiment Reversionspendel&Lilith Diringer&Praktikum Reversionspendel&6\\\hline \end{tabular}\\\vspace{5.00000mm}~\\\textbf{{\large{}Samstag 14:20 bis 15:50}}\nopagebreak \\\begin{tabular} {|p{5cm}|p{4cm}|p{4cm}|p{2cm}|}\hline \textbf{Thema}&\textbf{Betreuer}&\textbf{Raum}&\textbf{Anzahl}\\\hline Bestimmung des Brechungskoeffizienten von Plexiglas&Lilith Diringer&MW 0234&8\\\hline Gravitationsbeschleunigung&Ann-Kathrin Raab&MW 1050&8\\\hline Rotationsbewegungen&Vincent Grande&MW 0250&5\\\hline Thermodynamik 1&Maximilian Marienhagen&MW 1250&7\\\hline Elektrodynamik 2&Maximilian Keitel&MW 2050&4\\\hline Harmonische Schwingungen&Ilja Göthel&CH 26411&3\\\hline Elektrische Schaltungen&Felix Wechsler&CH 26410&3\\\hline Relativistische Teilchenphysik&Lars Dehlwes&CH 22210&25\\\hline Quanten- und Atomphysik I&Vitaly Andreev&CH 22209&23\\\hline \end{tabular}\\\vspace{5.00000mm}~\\\textbf{{\large{}Samstag 16:10 bis 17:40}}\nopagebreak \\\begin{tabular} {|p{5cm}|p{4cm}|p{4cm}|p{2cm}|}\hline \textbf{Thema}&\textbf{Betreuer}&\textbf{Raum}&\textbf{Anzahl}\\\hline Bestimmung des Brechungskoeffizienten von Wasser&Lilith Diringer&MW 0234&8\\\hline Komplexe Wechselstromrechnung&Vincent Grande&MW 0250&11\\\hline Elektrische Blackboxen&Eugen Dizer&Praktikum Blackboxen&6\\\hline Aufgabenseminar Elektrodynamik&Maximilian Keitel&MW 1250&6\\\hline Aufgabenseminar klassische Mechanik&Aaron Wild&MW 2050&6\\\hline Himmelsmechanik&Lars Dehlwes&CH 22210&9\\\hline Wellenoptik&Christopher Pfeiffer&CH 26410&12\\\hline Quanten- und Atomphysik II&Vitaly Andreev&CH 22209&28\\\hline \end{tabular}\\\vspace{5.00000mm}~\\\textbf{{\large{}Sonntag 09:00 bis 10:30}}\nopagebreak \\\begin{tabular} {|p{5cm}|p{4cm}|p{4cm}|p{2cm}|}\hline \textbf{Thema}&\textbf{Betreuer}&\textbf{Raum}&\textbf{Anzahl}\\\hline Gravitationsbeschleunigung&Ann-Kathrin Raab&MW 1050&8\\\hline Aufgabenseminar Elektrodynamik&Maximilian Keitel&MW 0234&6\\\hline Aufgabenseminar Wärmelehre&Maximilian Marienhagen&CH 22209&7\\\hline Theoretische Mechanik&Eugen Dizer&CH 22210&18\\\hline Wellenoptik&Christopher Pfeiffer&MW 2050&9\\\hline Quanten- und Atomphysik II&Vitaly Andreev&MW 1250&10\\\hline Elektronik&Martin Großhauser&CH 26410&13\\\hline Spezielle Relativitätstheorie&Johannes Rothe&MW 0250&15\\\hline \end{tabular}\\\vspace{5.00000mm}~\\\textbf{{\large{}Sonntag 10:50 bis 12:20}}\nopagebreak \\\begin{tabular} {|p{5cm}|p{4cm}|p{4cm}|p{2cm}|}\hline \textbf{Thema}&\textbf{Betreuer}&\textbf{Raum}&\textbf{Anzahl}\\\hline Bestimmung des Brechungskoeffizienten von Plexiglas&Lilith Diringer&MW 0234&8\\\hline Rotationsbewegungen&Vincent Grande&CH 26410&8\\\hline Elektrische Blackboxen&Eugen Dizer&Praktikum Blackboxen&6\\\hline Elektrodynamik 2&Maximilian Keitel&MW 2050&8\\\hline Relativistische Teilchenphysik&Lars Dehlwes&CH 22210&8\\\hline Aufgabenseminar Quanten- und Atomphysik und Struktur der Materie&Vitaly Andreev&MW 1250&25\\\hline Wellenoptik&Christopher Pfeiffer&MW 0250&13\\\hline Aufgabenseminar klassische Mechanik&Aaron Wild&CH 22209&10\\\hline \end{tabular}\\\vspace{5.00000mm}~\\\end{document}